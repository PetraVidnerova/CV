%%%%%%%%%%%%%%%%%%%%%%%%%%%%%%%%%%%%%%%%%
% Stylish Curriculum Vitae
% LaTeX Template
% Version 1.0 (18/7/12)
%
% This template has been downloaded from:
% http://www.LaTeXTemplates.com
%
% Original author:
% Stefano (http://stefano.italians.nl/)
%
% IMPORTANT: THIS TEMPLATE NEEDS TO BE COMPILED WITH XeLaTeX
%
% License:
% CC BY-NC-SA 3.0 (http://creativecommons.org/licenses/by-nc-sa/3.0/)
%
% The main font used in this template, Adobe Garamond Pro, does not 
% come with Windows by default. You will need to download it in
% order to get an output as in the preview PDF. Otherwise, change this 
% font to one that does come with Windows or comment out the font line 
% to use the default LaTeX font.
%
%%%%%%%%%%%%%%%%%%%%%%%%%%%%%%%%%%%%%%%%%

\documentclass[a4paper, oneside, final]{scrartcl} % Paper options using the scrartcl class

\usepackage{scrpage2} % Provides headers and footers configuration
\usepackage{titlesec} % Allows creating custom \section's
\usepackage{marvosym} % Allows the use of symbols
\usepackage{tabularx,colortbl} % Advanced table configurations
\usepackage{fontspec} % Allows font customization

\usepackage{hyperref}

\hypersetup{
    colorlinks=true,
    linkcolor=blue,
    filecolor=blue,      
    urlcolor=blue}
 
\urlstyle{same}


\defaultfontfeatures{Mapping=tex-text}
%\setmainfont{Adobe Garamond Pro} % Main document font

\titleformat{\section}{\large\scshape\raggedright}{}{0em}{}[\titlerule] % Section formatting

\pagestyle{scrheadings} % Print the headers and footers on all pages

\addtolength{\voffset}{-0.5in} % Adjust the vertical offset - less whitespace at the top of the page
\addtolength{\textheight}{3.5cm} % Adjust the text height - less whitespace at the bottom of the page

\newcommand{\gray}{\rowcolor[gray]{.90}} % Custom highlighting for the work experience and education sections
%----------------------------------------------------------------------------------------
% ÊFOOTER SECTION
%----------------------------------------------------------------------------------------

\renewcommand{\headfont}{\normalfont\rmfamily\scshape} % Font settings for footer

%% \cofoot{
%% \addfontfeature{LetterSpace=20.0}\fontsize{12.5}{17}\selectfont % Letter spacing and font size%

%% 123 Broadway {\large\textperiodcentered} City {\large\textperiodcentered} Country 12345\\ % Your mailing address
%% {\Large\Letter} john@smith.com \ {\Large\Telefon} (000) 111-1111 % Your email address and phone number
%% }

%----------------------------------------------------------------------------------------

\begin{document}

%\begin{center} % Center everything in the document

%----------------------------------------------------------------------------------------
% ÊHEADER SECTION
%----------------------------------------------------------------------------------------

%{\scshape\fontsize{18}{18} Curriculum Vitae}
\noindent
{\addfontfeature{LetterSpace=20.0}\fontsize{18}{18}\selectfont\scshape Curriculum Vitae} 

%\vspace{1.5cm} % Extra whitespace after the large name at the top


\section{Personal Data}
\begin{tabularx}{0.97\linewidth}{>{\raggedright}p{2cm}X}
Full Name & 	Petra Vidnerov\'a, n\'ee Kudov\'a\\
Born & 7 May 1977 in Plze\v{n}, Czech Republic\\
Citizenship 	& Czech Republic\\
Contact & petra@cs.cas.cz, \href{http://www.cs.cas.cz/petra}{http://www.cs.cas.cz/petra}
\end{tabularx}



\section{Research Interests}

Machine learning, supervised learning, kernel methods, regularization networks.
Deep learning. Hyper-parameter setup, meta-learning. Neural architecture search.

\noindent
Genetic algorithms, evolutionary and hybrid approaches.

\noindent
Epidemic modelling.

%----------------------------------------------------------------------------------------
%	WORK EXPERIENCE
%----------------------------------------------------------------------------------------

\section{Work Experience}

\begin{tabularx}{0.97\linewidth}{>{\raggedright}p{2cm}X}
  \gray since 2001 & \textbf{scientist},   Institute of Computer science, \\
  \gray & \hspace{1.6cm} The Czech Academy of Sciences \\
  \gray & Department of Machine Learning.\\
%  & \\
  & 2007-2012 \textbf{postdoc}, 2001-2012 \textbf{PhD student}, \\
  & 2008-2015 working part-time (maternity leave). \\
\end{tabularx}

\vspace{12pt}

%----------------------------------------------------------------------------------------
%	EDUCATION
%----------------------------------------------------------------------------------------

\section{Education}

\begin{tabularx}{0.97\linewidth}{>{\raggedleft}p{2cm}X}
\gray 2007 & Ph.D., Faculty of Mathematics and Physics, Charles University.\\
 & PhD thesis: {\em Learning with Regularization Networks.} \\
& \\
\gray 2003 & RNDr.  Faculty of Mathematics and Physics, Charles University.\\
%% \gray  & Faculty of Mathematics and Physics,\\
%% \gray & Charles University, Prague.\\
 & \\
\gray 2001 & Mgr. in Computer Science, \\
\gray  & Faculty of Mathematics and Physics,  Charles University, Prague.\\
& Master thesis: {\em  Learning algorithms for RBF networks.} \\ %Supervised by Mgr. Roman Neruda, CSc. \\
%% & Software project: MAGDON (Data Mining using genetic algorithms).\\
%% & During study focused on neural networks and computer graphics. \\
\end{tabularx}

\vspace{12pt}


\section{Visits Abroad}

\begin{tabularx}{0.97\linewidth}{>{\raggedleft}p{2cm}X}
  \gray February 2006 & Machine Learning Summer School. Canberra, Australia.  \\
  \\
\gray April - June 2005,  
November 2005 	& Two visits at Edinburgh Parallel Computing Center (EPCC), Edinburgh University, United Kingdom.  As a grantee of HPC-Europa project.\\
& Hosted by Prof. Ben Paechter, School of Computing, Napier University, Edinburgh. \\
\\
\gray  July 2002 & 	Neural Networks Summer School. Porto, Portugal.\\
\end{tabularx}

%----------------------------------------------------------------------------------------

\section{Awards}
\begin{tabularx}{0.97\linewidth}{>{\raggedleft}p{2cm}X}
  Best Paper Award & conference ITAT, Slovakia, 2017, P. Vidnerov\'a, R. Neruda. Evolution Strategies for Deep Neural Network Models Design. \\ 
\end{tabularx}

\section{Teaching}
\begin{tabularx}{0.97\linewidth}{>{\raggedleft}p{2cm}X}
\gray Courses &   Evolutionary algorithms (practical course), The Faculty of Mathematics and Physics, Charles University,
2006-2008 \\
\\
\gray Students & Rudolf Kadlec, The Faculty of Mathematics and Physics, Charles University \\
&  supervising Rudolf's diploma thesis: Evolution of intelligent agent behaviour in computer games,
    2008 \\
\\
\gray Commitee Memberships & 
    the opponent of Ing. Martin \v{S}lap\'ak's thesis,
    Czech Technical University (2018, 2019),
     the opponent of RNDr. Viliam Dillinger's thesis, Comenius University in Bratislava, (2019). \\
\end{tabularx}

%% \section{Current Grant Projects}
%%  %% Capabilities and Limitations of Shallow and Deep Networks,
%%  %% Czech Grant Agency, no. 18-23827S, 2018-2020 (team member)
%%  National Competence Center - Cybernetics and Artificial Intelligence,
%%  Technology Agency of the Czech Republic, no. TN01000024, 2019 - 2022 (team member) 
%%  \newline
%%  AppNeCo: Approximate Neurocomputing,
%%  Czech Grant Agency, no. 22-02067S, 2022-2024 (team member)
%%  %% Město pro lidi, ne pro virus - Technology Agency of the Czech Republic, no. TL04000282, 2020/21 (team member) 


 \section{Selected Publications}
 \noindent
J. Kalina, A. Neoral, P. Vidnerová.
{\em Effective Automatic Method Selection for Nonlinear Regression Modeling.}
International Journal of Neural Systems. Roč. 31, č. 10 (2021), paper no. 2150020. ISSN 0129-0657.

\noindent
P. Vidnerov\'a, R. Neruda.  {\em Vulnerability of classifiers to evolutionary
  generated adversarial examples.}  Neural Networks. Volume 127, July 2020,
p. 168-181. ISSN 0893-6080.

\noindent
S. Slu\v{s}n\'y, R. Neruda, P. Vidnerov\'a.  {\em Comparison of
  Behavior-based and Planning Techniques on the Small Robot Maze
  Exploration Problem.}  Neural Networks. Volume 23, Issue 4 (2010),
p. 560-567. ISSN 0893-6080.

\noindent
R. Neruda, P. Kudov\'a.  {\em Learning Methods for Radial Basis
  Functions Networks.}  Future Generation Computer
Systems. 21. (2005), p. 1131-1142. ISSN 0167-739X

\section{Software}
\begin{tabularx}{0.97\linewidth}{>{\raggedleft}p{2cm}X}
\gray  rbf\_keras & Implementation of an RBF layer for the Keras library. \\
   & 
  \href{https://github.com/PetraVidnerova/rbf_keras}{https://github.com/PetraVidnerova/rbf\_keras}
  (used by scientific community, 4 citations)\\
  \\
  \gray Model M & Multiagent epidemic model. One of key developers. \\
  & \href{https://github.com/epicity-cz/model-m}{https://github.com/epicity-cz/model-m} \\
\end{tabularx}


%% \section{Community Service}

%% \begin{tabularx}{0.97\linewidth}{>{\raggedleft}p{2cm}X}
%% \gray  professional & PC member, reviewer \\
%%     & member of conference programme committees:
%%     AIAI 2016, AIAI 2018-2021, EANN 2015-2021, EML GECCO 2016-2021, IJCNN 2017, IJCNN 2019, ICANN 2018, ITAT 2009 \\
%%     & reviewing for scientific journals: Neural Processing Letters,
%%     IEEE Transactions on Cybernetics, Computing and Informatics, IEEE
%%     Transactions on Evolutionary Computations, Neural Networks, Natural
%%     Computing, Analytical Letters, IEEE Transactions on Neural Networks
%%     and Learning Systems, Computer Science Review, IEEE Sensors Journal,
%%     Computers \& security \\
%%     & reviewer for GA UK \\
%%     \\
%% %\gray other & \\
%%     \gray & taking care of blog of Institute of Computer Science (since 2015)\\
%%      & \href{http://zatisi.cs.cas.cz}{http://zatisi.cs.cas.cz} \\
%%     \\
%%      \gray & BISOP, scientific board member (since 2020) \\
%%      & \href{http://bisop.cz}{http://bisop.cz} \\
%%      \\
%% \gray free-time & teaching at PyLadies.cz courses (since 2018) \\
%% & PyLadies is a community of female Python programmers helping women to get
%% familiar with IT. \\
%% \gray & author of machine learning study materials for   
%% data analysis course organised by PyLadies \&  PyData community (2020).
%% \end{tabularx}


\section{Languages}
\begin{tabularx}{0.97\linewidth}{>{\raggedleft}p{2cm}X}
  %% Czech &  native \\
  English & C1 (CAE certificate, 2006) \\
  German & elementary \\ 
\end{tabularx}

\section{Other skills}
\begin{tabularx}{0.97\linewidth}{>{\raggedleft}p{2.5cm}X}
  \gray  Programming Languages &  Python, bash (in past: Pascal, C/C++, MPI, Perl, PHP, SQL, JavaScript),
   basic knowledge of HTML and CSS\\
   %% & familiar with Python libraries: numpy, pandas, matplotlib, seaborn, scikit-learn,
   %% Keras, Tensorflow\\
   %% & moderate experience with Django framework \\
   %% & \href{http://www.cs.cas.cz/~petra/cv/cert/certificate_intel.pdf}{AI Intel certificate}\\
  Other & LaTeX, git, enthusiastic Linux user \\
\end{tabularx}


%% \section{Miscellaneous}
%% {\small
%% \begin{tabularx}{0.97\linewidth}{>{\raggedleft}p{2.5cm}X}
%%   certificates & First Aid Kids - Basic Life Support. SHOCK Training Center. (2011) \\
%%   & {\em Respektovat a b\'yt respektov\'an.} Course of communication
%%   skills focused on a parent-child communication. Accredited by M\v{S}MT (Ministry of Education, Youth and Sports). (2013) \\
%%   past experience & Leadership of programming hobby courses for secondary school children (Pascal, C). (1996-2001) \\
%% \end{tabularx}
%% }

\end{document}
