%%%%%%%%%%%%%%%%%%%%%%%%%%%%%%%%%%%%%%%%%
% Stylish Curriculum Vitae
% LaTeX Template
% Version 1.0 (18/7/12)
%
% This template has been downloaded from:
% http://www.LaTeXTemplates.com
%
% Original author:
% Stefano (http://stefano.italians.nl/)
%
% IMPORTANT: THIS TEMPLATE NEEDS TO BE COMPILED WITH XeLaTeX
%
% License:
% CC BY-NC-SA 3.0 (http://creativecommons.org/licenses/by-nc-sa/3.0/)
%
% The main font used in this template, Adobe Garamond Pro, does not 
% come with Windows by default. You will need to download it in
% order to get an output as in the preview PDF. Otherwise, change this 
% font to one that does come with Windows or comment out the font line 
% to use the default LaTeX font.
%
%%%%%%%%%%%%%%%%%%%%%%%%%%%%%%%%%%%%%%%%%

\documentclass[a4paper, oneside, final]{scrartcl} % Paper options using the scrartcl class

\usepackage{scrlayer-scrpage} % Provides headers and footers configuration
\usepackage{titlesec} % Allows creating custom \section's
\usepackage{marvosym} % Allows the use of symbols
\usepackage{tabularx,colortbl} % Advanced table configurations
\usepackage{fontspec} % Allows font customization

\usepackage{hyperref}

\hypersetup{
    colorlinks=true,
    linkcolor=blue,
    filecolor=blue,      
    urlcolor=blue}
 
\urlstyle{same}


\defaultfontfeatures{Mapping=tex-text}
%\setmainfont{Adobe Garamond Pro} % Main document font

\titleformat{\section}{\large\scshape\raggedright}{}{0em}{}[\titlerule] % Section formatting

\pagestyle{scrheadings} % Print the headers and footers on all pages

\addtolength{\voffset}{-0.5in} % Adjust the vertical offset - less whitespace at the top of the page
\addtolength{\textheight}{3.5cm} % Adjust the text height - less whitespace at the bottom of the page

\newcommand{\gray}{\rowcolor[gray]{.90}} % Custom highlighting for the work experience and education sections

%----------------------------------------------------------------------------------------
% ÊFOOTER SECTION
%----------------------------------------------------------------------------------------

\renewcommand{\headfont}{\normalfont\rmfamily\scshape} % Font settings for footer

%% \cofoot{
%% \addfontfeature{LetterSpace=20.0}\fontsize{12.5}{17}\selectfont % Letter spacing and font size%

%% 123 Broadway {\large\textperiodcentered} City {\large\textperiodcentered} Country 12345\\ % Your mailing address
%% {\Large\Letter} john@smith.com \ {\Large\Telefon} (000) 111-1111 % Your email address and phone number
%% }

%----------------------------------------------------------------------------------------

\begin{document}

%\begin{center} % Center everything in the document

%----------------------------------------------------------------------------------------
% ÊHEADER SECTION
%----------------------------------------------------------------------------------------

%{\scshape\fontsize{18}{18} Curriculum Vitae}
\noindent
{\addfontfeature{LetterSpace=20.0}\fontsize{18}{18}\selectfont\scshape Curriculum Vitae} 

%\vspace{1.5cm} % Extra whitespace after the large name at the top


\section{Personal Data}
\begin{tabularx}{0.97\linewidth}{>{\raggedright}p{2cm}X}
Full Name & 	Petra Vidnerov\'a, n\'ee Kudov\'a\\
Born & 7 May 1977 in Plze\v{n}, Czech Republic\\
Citizenship 	& Czech Republic\\
Contact & petra@cs.cas.cz, \href{http://www.cs.cas.cz/petra}{http://www.cs.cas.cz/petra}\\
\end{tabularx}

\centerline{}
\centerline{
  ORCID: \href{https://orcid.org/0000-0003-3879-3459}{0000-0003-3879-3459}
  ResearchID: \href{https://www.webofscience.com/wos/author/record/G-2718-2014}{G-2718-2014}
  Scopus: \href{https://www.scopus.com/authid/detail.url?authorId=25121797400}{25121797400}
  \hfill}



\section{Research Interests}

Machine learning, supervised learning. Deep learning.

\noindent
Hyper-parameter setup, meta-learning. AutoML. Neural architecture search.

\noindent
Genetic algorithms, evolutionary and hybrid approaches.

\noindent
Epidemic modelling. Agent based models.

%----------------------------------------------------------------------------------------
%	WORK EXPERIENCE
%----------------------------------------------------------------------------------------

\section{Work Experience}

\begin{tabularx}{0.97\linewidth}{>{\raggedright}p{2cm}X}
  \gray since 2012 & \textbf{scientist},   Institute of Computer science, \\
  \gray & \hspace{1.6cm} The Czech Academy of Sciences \\
  & Department of Artificial Intelligence (in the past Department of Machine Learning, Department of Theoretical Computer Science).\\
  & \\
  \gray 2007 - 2012 & \textbf{postdoc}, Institute of Computer science, \\
  \gray & \hspace{1.5cm} The Czech Academy of Sciences \\
  & Mainly working part time (parental leave).\\
  &\\
  \gray 2001 - 2007 & \textbf{PhD student},   Institute of Computer science, \\
  \gray & \hspace{2.5cm} The Czech Academy of Sciences \\
  &  One of the key developers of the multi-agent system Bang (system designed for hybrid models of artificial intelligence, written in C/C++). \\
\end{tabularx}

\vspace{12pt}

%----------------------------------------------------------------------------------------
%	EDUCATION
%----------------------------------------------------------------------------------------

\section{Education}

\begin{tabularx}{0.97\linewidth}{>{\raggedleft}p{2cm}X}
\gray 2001 - 2007 & PhD at Faculty of Mathematics and Physics,\\
\gray & Charles University, Prague.\\
 & Topic of PhD thesis: {\em Learning with Regularization Networks.} Supervised by Mgr. Roman Neruda, CSc.\\
& \\ 
\gray 2003 & RNDr. in Computer Science, \\
\gray  & Faculty of Mathematics and Physics,\\
\gray & Charles University, Prague.\\
 & \\
\gray 1995 - 2001 & Mgr. in Computer Science, \\
\gray  & Faculty of Mathematics and Physics,\\
\gray & Charles University, Prague.\\
& Master thesis: {\em  Learning algorithms for RBF networks.} Supervised by Mgr. Roman Neruda, CSc. \\
& Software project: MAGDON (Data mining using genetic algorithms).\\
% & During study focused on neural networks and computer graphics. \\
\end{tabularx}

\vspace{12pt}


\section{Visits Abroad}

\begin{tabularx}{0.97\linewidth}{>{\raggedleft}p{2cm}X}
  \gray  February 2006 & Machine Learning Summer School. Canberra, Australia. (Volunteering.) \\
 & \\
\gray April - June 2005,  
November 2005 	& Two visits at Edinburgh Parallel Computing Center (EPCC), Edinburgh University, United Kingdom. \\
& As a grantee of HPC-Europa project. Hosted by Prof. Ben Paechter, School of Computing, Napier University, Edinburgh. \\
& \\
\gray  July 2002 & 	Neural Networks Summer School. Porto, Portugal.
\end{tabularx}

%----------------------------------------------------------------------------------------

\section{Awards}
\begin{tabularx}{0.97\linewidth}{>{\raggedleft}p{2cm}X}
  \gray  Best Paper Award & conference ITAT, Slovakia, 2017, P. Vidnerov\'a, R. Neruda. Evolution Strategies for Deep Neural Network Models Design. \\
  &\\
  \gray  Best Result of ICS & for the year 2022,  in the cathegory {\em Publication with Application or Social Impact}\\
  & awarded paper:
 { L. Berec, R. Levínský, J. Weiner, M. Šmíd, R. Neruda, P. Vidnerová, G. Suchopárová: Importance of vaccine action and availability and epidemic severity for delaying the second vaccine dose. Scientific Reports, 2022}   
\end{tabularx}



\section{Teaching and Comittee Memberships}
\begin{tabularx}{0.97\linewidth}{>{\raggedleft}p{2cm}X}
\gray Courses &   Evolutionary algorithms (practical course), The Faculty of Mathematics and Physics, Charles University,
2006-2008 \\
& \\
\gray Students & Rudolf Kadlec, The Faculty of Mathematics and Physics, Charles University \\
&  supervising Rudolf's diploma thesis: Evolution of intelligent agent behaviour in computer games,
    2008 \\
& \\
\gray Commitee Memberships & 
    committee for PhD thesis defence, the opponent of Ing. Martin \v{S}lap\'ak's thesis,
    Faculty of Information Technology, Czech Technical University (2018, 2019) \\
 & committee for PhD thesis defence, the opponent of RNDr. Viliam Dillinger's thesis,
Comenius University in Bratislava (2019) \\
\gray & committee for PhD thesis defence, the opponent of Ing. Dalibor Cimr's thesis,
University of Hradec Králové, Faculty of Informatics and Management (2023) 
\end{tabularx}

\section{Current  Projects}
 %% Capabilities and Limitations of Shallow and Deep Networks,
 %% Czech Grant Agency, no. 18-23827S, 2018-2020 (team member)
 %% National Competence Center - Cybernetics and Artificial Intelligence,
 %% Technology Agency of the Czech Republic, no. TN01000024, 2019 - 2022 (team member) 
 %% \newline
 AppNeCo: Approximate Neurocomputing,
 Czech Grant Agency, no. 22-02067S, 2022-2024 (team member)
 %% Město pro lidi, ne pro virus - Technology Agency of the Czech Republic, no. TL04000282, 2020/21 (team member) 

\section{Recent  Projects}
 \noindent
 National Competence Center - Cybernetics and Artificial Intelligence,
 Technology Agency of the Czech Republic, no. TN01000024, 2019 - 2022 (team member) 

 \noindent
 Město pro lidi, ne pro virus - Technology Agency of the Czech Republic, no. TL04000282, 2020/21 (team member) 

\noindent
Capabilities and Limitations of Shallow and Deep Networks,
Czech Grant Agency,\newline no. 18-23827S, 2018-2020 (team member)


\noindent
Model complexity of neural, radial, and kernel networks,
Czech Grant Agency, \newline no. 15-18108S, 	2015-2017 (team member) 



 \section{Selected Publications}
 \noindent
L. Berec, T. Diviák, A. Kuběna, R. Levínský, R. Neruda, G. Suchopárová, J. Šlerka, M. Šmíd, J. Trnka, V. Tuček, Petra Vidnerová, M. Zajíček,
{\em On the contact tracing for COVID-19: A simulation study}, 
Epidemics, Volume 43, (2023), ISSN 1755-4365.

 \noindent
J. Kalina, A. Neoral, P. Vidnerová.
{\em Effective Automatic Method Selection for Nonlinear Regression Modeling.}
International Journal of Neural Systems. Roč. 31, č. 10 (2021), paper no. 2150020. ISSN 0129-0657.

\noindent
P. Vidnerov\'a, R. Neruda.  {\em Vulnerability of classifiers to evolutionary
  generated adversarial examples.}  Neural Networks. Volume 127, July 2020,
p. 168-181. ISSN 0893-6080.

\noindent
S. Slu\v{s}n\'y, R. Neruda, P. Vidnerov\'a.  {\em Comparison of
  Behavior-based and Planning Techniques on the Small Robot Maze
  Exploration Problem.}  Neural Networks. Volume 23, Issue 4 (2010),
p. 560-567. ISSN 0893-6080.

\noindent
R. Neruda, P. Kudov\'a.  {\em Learning Methods for Radial Basis
  Functions Networks.}  Future Generation Computer
Systems. 21. (2005), p. 1131-1142. ISSN 0167-739X

\section{Software}
\begin{tabularx}{0.97\linewidth}{>{\raggedleft}p{2cm}X}
  rbf\_keras & Implementation of an RBF layer for the Keras library. \\
   & Available at
  \href{https://github.com/PetraVidnerova/rbf_keras}{https://github.com/PetraVidnerova/rbf\_keras}\\
  & (12 citations accoding to GoogleScholar, 136 Github stars)\\
&  \\
  Model M & Multiagent epidemic model. One of the key developers. \\
  & Available at
  \href{https://github.com/epicity-cz/model-m}{https://github.com/epicity-cz/model-m} 

\end{tabularx}

\section{Selected Talks}
    {\em From perceptron to deep neural networks}, 2019,
    Workshop Teorie a praxe statistického zpracování dat,
    Palacký University Olomouc,  Nová Seninka.

\noindent
    {\em Adversarial examples - vulnerability of machine learning
      methods and prevention}, 2018, Seminar of the Institute of
    Information Theory and Automation of the Czech Academy of
    Sciences, Prague.

\noindent
    {\em Evolving Architectures of Deep Neural Networks}, 2018,
 Machine Learning and Modelling Seminar, The Faculty
of Mathematics and Physics, Charles University, Prague.

\noindent
{\em Evolution of Composite Kernel Functions for Regularization
  Networks}, 2011, Machine Learning and Modelling Seminar, The Faculty
of Mathematics and Physics, Charles University, Prague.

\noindent
{\em Hybrid learning methods in Bang and Regularization Networks},
 2005, department seminar at University of Edinburgh, UK.

\section{Popularization}
Talk {\em Model M - an agent based epidemiological model}, at the
BISOP book launch event, 2023.

\noindent
Talk in Czech {\em Umělá inteligence: dobrý sluha, zlý pán?}, Open
Day, Institute of Computer Science, The Czech Academy of Sciences,
2019.

\noindent
Talk in Czech {\em Hluboké neuronové sítě}, Open
Day, Institute of Computer Science, The Czech Academy of Sciences,
2017.

\noindent
Joint talk with Roman Neruda at the seminar for high school teachers,
Nové Hrady, 2008.

\section{Community Service}

\begin{tabularx}{0.97\linewidth}{>{\raggedleft}p{2cm}X}
\gray  professional & PC member, reviewer \\
    & member of conference programme committees:
 AIAI 2016, AIAI 2018-2023, EANN 2015-2023, EML GECCO 2016-2023, IJCNN 2017, IJCNN 2019-2023,
ICANN 2018, ICANN 2023, ICONIP 2023, ITAT 2009 \\
    & reviewing for scientific journals: Neural Processing Letters,
    IEEE Transactions on Cybernetics, Computing and Informatics, IEEE
    Transactions on Evolutionary Computations, Neural Networks, Natural
    Computing, Analytical Letters, IEEE Transactions on Neural Networks
    and Learning Systems, Computer Science Review, IEEE Sensors Journal,
    Computers \& security; reviewer for GA UK \\
%\gray other & \\
  &  \\
    \gray & working as a Scientific Secretary of  Institute of Computer Science (since 2023)\\
    & taking care of the \href{http://zatisi.cs.cas.cz}{blog} of Institute of Computer Science (since 2015) \\

    &  \\

    \gray & BISOP, scientific board member (since 2020) \\
     & \href{http://bisop.cz}{http://bisop.cz} \\
  &   \\
\gray free-time & teaching at PyLadies.cz courses (since 2018) \\
& PyLadies is a community of female Python programmers helping women to get
familiar with IT. \\
\gray & author of machine learning study materials for   
data analysis course organised by PyLadies \&  PyData community (2020).
\end{tabularx}


\section{Languages}
\begin{tabularx}{0.97\linewidth}{>{\raggedleft}p{2cm}X}
  Czech &  native \\
  English & C1 (CAE certificate, 2006) \\
  German & elementary \\ 
\end{tabularx}

\section{Other skills}
\begin{tabularx}{0.97\linewidth}{>{\raggedleft}p{2.5cm}X}
  \gray  Programming Languages &  Python, bash (in past: Pascal, C/C++, MPI, Perl, PHP, SQL, JavaScript),
   basic knowledge of HTML and CSS\\
   & familiar with Python libraries: numpy, pandas, matplotlib, seaborn, scikit-learn,
   Keras, Tensorflow, Pytorch\\
%   & moderate experience with Django framework \\
   & \href{http://www.cs.cas.cz/~petra/cv/cert/certificate_intel.pdf}{AI Intel certificate}\\
\gray  Other & LaTeX, git, enthusiastic Linux user \\
%\gray   & enthusiastic Linux user \\
\end{tabularx}


%% \section{Miscellaneous}
%% {\small
%% \begin{tabularx}{0.97\linewidth}{>{\raggedleft}p{2.5cm}X}
%%   certificates & First Aid Kids - Basic Life Support. SHOCK Training Center. (2011) \\
%%   & {\em Respektovat a b\'yt respektov\'an.} Course of communication
%%   skills focused on a parent-child communication. Accredited by M\v{S}MT (Ministry of Education, Youth and Sports). (2013) \\
%%   past experience & Leadership of programming hobby courses for secondary school children (Pascal, C). (1996-2001) \\
%% \end{tabularx}
%% }

\pagebreak

\noindent
{\addfontfeature{LetterSpace=20.0}\fontsize{18}{18}\selectfont\scshape Full List of Publications} 

\section{Journal papers}


\section{Conference proceedings}

\section{Other}

\section{Software}


\pagebreak

\noindent
{\addfontfeature{LetterSpace=20.0}\fontsize{18}{18}\selectfont\scshape Description of selected results} 

\section{Learning Algorithms for RBF Networks}
In the 1990s and early 21st century, the field of machine learning
was dominated by the so-called kernel methods. Among these, we can
include the highly popular Radial Basis Function Networks (RBF
networks).

As part of my doctoral thesis, I delved into these methods. One of the
primary outcomes [1] was the proposal of several algorithms for
training RBF networks and an analysis of these algorithms. All methods
were implemented and experiments were conducted to demonstrate the
relationship between the training time and the accuracy of the
results.

\centerline{}
\noindent
[1] R. Neruda, P. Kudová. Learning Methods for Radial Basis Functions
Networks. Future Generation Computer Systems. 21. (2005),
p. 1131-1142. ISSN 0167-739X
\href{https://dl.acm.org/doi/10.5555/1088377.1708275}{https://dl.acm.org/doi/10.5555/1088377.1708275}


\section{Generating of Adversarial Examples and Their Analysis}

In recent years, the question of the security and reliability of
artificial intelligence algorithms has gained significant
attention. This is closely related to the concept of "adversarial"
patterns, which are intentionally created patterns designed to deceive
a given classification model. For instance, in image classification,
an adversarial pattern may appear indistinguishable from the original
image to the human eye, yet a properly trained model may classify it
incorrectly.

In our article [2], we have contributed to the understanding
of adversarial patterns, particularly emphasizing that they are not
limited to neural networks but apply to a wide range of classifiers,
from SVMs to decision trees. To address this issue, we have introduced
our own evolutionary algorithm for adversarial
patterns crafting. Its notable feature is its "black-box" approach, meaning it
doesn't require knowledge of the model's internal settings and
structure. As a result, it can be employed to undermine (in the
context of adversarial patterns) virtually any classification model.

Furthermore, through our experiments, we have demonstrated that
certain adversarial patterns can be transferred between models,
particularly when dealing with models of similar nature (e.g., two
neural networks with similar architectures).

Additionally, we have shown that networks equipped with local units
(e.g., RBF networks) exhibit higher resilience against adversarial
patterns.

\centerline{}
\noindent
[2] P. Vidnerová, R. Neruda. Vulnerability of classifiers to
evolutionary generated adversarial examples, Neural Networks, Volume
127, 2020.\newline
\href{https://doi.org/10.1016/j.neunet.2020.04.015}{https://doi.org/10.1016/j.neunet.2020.04.015}


\section{Epidemic Modeling - Multi-agent System Model M}
In 2020, I joined a group of scientists focused on modeling the spread
of the COVID-19 disease. One of the key outcomes is a multi-agent
model named Model M [3], along with software [4] that enables
experimentation with the model.

The model operates with a realistic network graph and allows for
simulating a wide range of interventions, such as global contact
restrictions, quarantines, and contact tracing.

The model is designed modularly, allowing for diverse application
deployments. The model was used in a study on virus transmission in
schools in collaboration with the Ministry of Education, Youth, and
Sports of the Czech Republic.

I contributed to the model's design, implementation, and experiments.

\centerline{}
\noindent
[3] Berec, et al. On the Contact Tracing for COVID-19: A simulation
study. Epidemics, Volume 43, (2023), ISSN 1755-4365.
\href{https://doi.org/10.1016/j.epidem.2023.100677}{https://doi.org/10.1016/j.epidem.2023.100677}

\noindent
[4] Berec, et al. {\em Epicity}, 2021,
\href{https://github.com/epicity-cz/model-m/releases/tag/v1.0}{https://github.com/epicity-cz/model-m/releases/tag/v1.0}


\end{document}
