%%%%%%%%%%%%%%%%%%%%%%%%%%%%%%%%%%%%%%%%%
% Stylish Curriculum Vitae
% LaTeX Template
% Version 1.0 (18/7/12)
%
% This template has been downloaded from:
% http://www.LaTeXTemplates.com
%
% Original author:
% Stefano (http://stefano.italians.nl/)
%
% IMPORTANT: THIS TEMPLATE NEEDS TO BE COMPILED WITH XeLaTeX
%
% License:
% CC BY-NC-SA 3.0 (http://creativecommons.org/licenses/by-nc-sa/3.0/)
%
% The main font used in this template, Adobe Garamond Pro, does not 
% come with Windows by default. You will need to download it in
% order to get an output as in the preview PDF. Otherwise, change this 
% font to one that does come with Windows or comment out the font line 
% to use the default LaTeX font.
%
%%%%%%%%%%%%%%%%%%%%%%%%%%%%%%%%%%%%%%%%%

\documentclass[a4paper, oneside, final]{scrartcl} % Paper options using the scrartcl class

\usepackage{scrlayer-scrpage} % Provides headers and footers configuration
\usepackage{titlesec} % Allows creating custom \section's
\usepackage{marvosym} % Allows the use of symbols
\usepackage{tabularx,colortbl} % Advanced table configurations
\usepackage{fontspec} % Allows font customization

\usepackage{hyperref}

\hypersetup{
    colorlinks=true,
    linkcolor=blue,
    filecolor=blue,      
    urlcolor=blue}
 
\urlstyle{same}


\defaultfontfeatures{Mapping=tex-text}
%\setmainfont{Adobe Garamond Pro} % Main document font

\titleformat{\section}{\large\scshape\raggedright}{}{0em}{}[\titlerule] % Section formatting

\pagestyle{scrheadings} % Print the headers and footers on all pages

\addtolength{\voffset}{-0.5in} % Adjust the vertical offset - less whitespace at the top of the page
\addtolength{\textheight}{3.5cm} % Adjust the text height - less whitespace at the bottom of the page

\newcommand{\gray}{\rowcolor[gray]{.90}} % Custom highlighting for the work experience and education sections

%----------------------------------------------------------------------------------------
% ÊFOOTER SECTION
%----------------------------------------------------------------------------------------

\renewcommand{\headfont}{\normalfont\rmfamily\scshape} % Font settings for footer

%% \cofoot{
%% \addfontfeature{LetterSpace=20.0}\fontsize{12.5}{17}\selectfont % Letter spacing and font size%

%% 123 Broadway {\large\textperiodcentered} City {\large\textperiodcentered} Country 12345\\ % Your mailing address
%% {\Large\Letter} john@smith.com \ {\Large\Telefon} (000) 111-1111 % Your email address and phone number
%% }

%----------------------------------------------------------------------------------------

\begin{document}

%\begin{center} % Center everything in the document

%----------------------------------------------------------------------------------------
% ÊHEADER SECTION
%----------------------------------------------------------------------------------------

%{\scshape\fontsize{18}{18} Curriculum Vitae}
\noindent
{\addfontfeature{LetterSpace=20.0}\fontsize{18}{18}\selectfont\scshape Curriculum Vitae} 

%\vspace{1.5cm} % Extra whitespace after the large name at the top


\section{Personal Data}
\begin{tabularx}{0.97\linewidth}{>{\raggedright}p{2cm}X}
Full Name & 	Petra Vidnerov\'a, n\'ee Kudov\'a\\
Born & 7 May 1977 in Plze\v{n}, Czech Republic\\
Citizenship 	& Czech Republic\\
Contact & petra@cs.cas.cz, \href{http://www.cs.cas.cz/petra}{http://www.cs.cas.cz/petra}\\
\end{tabularx}

\centerline{}
\centerline{
  ORCID: \href{https://orcid.org/0000-0003-3879-3459}{0000-0003-3879-3459}
  ResearchID: \href{https://www.webofscience.com/wos/author/record/G-2718-2014}{G-2718-2014}
  Scopus: \href{https://www.scopus.com/authid/detail.url?authorId=25121797400}{25121797400}
  \hfill}



\section{Research Interests}

Machine learning, supervised learning. Deep learning.

\noindent
Hyper-parameter setup, meta-learning. AutoML. Neural architecture search.

\noindent
Genetic algorithms, evolutionary and hybrid approaches.

\noindent
Epidemic modelling. Agent based models.

%----------------------------------------------------------------------------------------
%	WORK EXPERIENCE
%----------------------------------------------------------------------------------------

\section{Work Experience}

\begin{tabularx}{0.97\linewidth}{>{\raggedright}p{2cm}X}
  \gray since 2012 & \textbf{scientist},   Institute of Computer science, \\
  \gray & \hspace{1.6cm} The Czech Academy of Sciences \\
  & Department of Artificial Intelligence (in the past Department of Machine Learning, Department of Theoretical Computer Science).\\
  & \\
  \gray 2007 - 2012 & \textbf{postdoc}, Institute of Computer science, \\
  \gray & \hspace{1.5cm} The Czech Academy of Sciences \\
  & Mainly working part time (parental leave).\\
  &\\
  \gray 2001 - 2007 & \textbf{PhD student},   Institute of Computer science, \\
  \gray & \hspace{2.5cm} The Czech Academy of Sciences \\
  &  One of the key developers of the multi-agent system Bang (system designed for hybrid models of artificial intelligence, written in C/C++). \\
\end{tabularx}

\vspace{12pt}

%----------------------------------------------------------------------------------------
%	EDUCATION
%----------------------------------------------------------------------------------------

\section{Education}

\begin{tabularx}{0.97\linewidth}{>{\raggedleft}p{2cm}X}
\gray 2001 - 2007 & PhD at Faculty of Mathematics and Physics,\\
\gray & Charles University, Prague.\\
 & Topic of PhD thesis: {\em Learning with Regularization Networks.} Supervised by Mgr. Roman Neruda, CSc.\\
& \\ 
\gray 2003 & RNDr. in Computer Science, \\
\gray  & Faculty of Mathematics and Physics,\\
\gray & Charles University, Prague.\\
 & \\
\gray 1995 - 2001 & Mgr. in Computer Science, \\
\gray  & Faculty of Mathematics and Physics,\\
\gray & Charles University, Prague.\\
& Master thesis: {\em  Learning algorithms for RBF networks.} Supervised by Mgr. Roman Neruda, CSc. \\
& Software project: MAGDON (Data mining using genetic algorithms).\\
% & During study focused on neural networks and computer graphics. \\
\end{tabularx}

\vspace{12pt}


\section{Visits Abroad}

\begin{tabularx}{0.97\linewidth}{>{\raggedleft}p{2cm}X}
  \gray  February 2006 & Machine Learning Summer School. Canberra, Australia. (Volunteering.) \\
 & \\
\gray April - June 2005,  
November 2005 	& Two visits at Edinburgh Parallel Computing Center (EPCC), Edinburgh University, United Kingdom. \\
& As a grantee of HPC-Europa project. Hosted by Prof. Ben Paechter, School of Computing, Napier University, Edinburgh. \\
& \\
\gray  July 2002 & 	Neural Networks Summer School. Porto, Portugal.
\end{tabularx}

%----------------------------------------------------------------------------------------

\section{Awards}
\begin{tabularx}{0.97\linewidth}{>{\raggedleft}p{2cm}X}
  \gray  Best Paper Award & conference ITAT, Slovakia, 2017, P. Vidnerov\'a, R. Neruda. Evolution Strategies for Deep Neural Network Models Design. \\
  &\\
  \gray  Best Result of ICS & for the year 2022,  in the cathegory {\em Publication with Application or Social Impact}\\
  & awarded paper:
 { L. Berec, R. Levínský, J. Weiner, M. Šmíd, R. Neruda, P. Vidnerová, G. Suchopárová: Importance of vaccine action and availability and epidemic severity for delaying the second vaccine dose. Scientific Reports, 2022}   
\end{tabularx}



\section{Teaching and Comittee Memberships}
\begin{tabularx}{0.97\linewidth}{>{\raggedleft}p{2cm}X}
\gray Courses &   Evolutionary algorithms (practical course), The Faculty of Mathematics and Physics, Charles University,
2006-2008 \\
& \\
\gray Students & Rudolf Kadlec, The Faculty of Mathematics and Physics, Charles University \\
&  supervising Rudolf's diploma thesis: Evolution of intelligent agent behaviour in computer games,
    2008 \\
& \\
\gray Commitee Memberships & 
    committee for PhD thesis defence, the opponent of Ing. Martin \v{S}lap\'ak's thesis,
    Faculty of Information Technology, Czech Technical University (2018, 2019) \\
 & committee for PhD thesis defence, the opponent of RNDr. Viliam Dillinger's thesis,
Comenius University in Bratislava (2019) \\
\gray & committee for PhD thesis defence, the opponent of Ing. Dalibor Cimr's thesis,
University of Hradec Králové, Faculty of Informatics and Management (2023) 
\end{tabularx}

\section{Current  Projects}
 %% Capabilities and Limitations of Shallow and Deep Networks,
 %% Czech Grant Agency, no. 18-23827S, 2018-2020 (team member)
 %% National Competence Center - Cybernetics and Artificial Intelligence,
 %% Technology Agency of the Czech Republic, no. TN01000024, 2019 - 2022 (team member) 
 %% \newline
 AppNeCo: Approximate Neurocomputing,
 Czech Grant Agency, no. 22-02067S, 2022-2024 (team member)
 %% Město pro lidi, ne pro virus - Technology Agency of the Czech Republic, no. TL04000282, 2020/21 (team member) 

\section{Recent  Projects}
 \noindent
 National Competence Center - Cybernetics and Artificial Intelligence,
 Technology Agency of the Czech Republic, no. TN01000024, 2019 - 2022 (team member) 

 \noindent
 Město pro lidi, ne pro virus - Technology Agency of the Czech Republic, no. TL04000282, 2020/21 (team member) 

\noindent
Capabilities and Limitations of Shallow and Deep Networks,
Czech Grant Agency,\newline no. 18-23827S, 2018-2020 (team member)


\noindent
Model complexity of neural, radial, and kernel networks,
Czech Grant Agency, \newline no. 15-18108S, 	2015-2017 (team member) 



 \section{Selected Publications}
 \noindent
L. Berec, T. Diviák, A. Kuběna, R. Levínský, R. Neruda, G. Suchopárová, J. Šlerka, M. Šmíd, J. Trnka, V. Tuček, Petra Vidnerová, M. Zajíček,
{\em On the contact tracing for COVID-19: A simulation study}, 
Epidemics, Volume 43, (2023), ISSN 1755-4365.

 \noindent
J. Kalina, A. Neoral, P. Vidnerová.
{\em Effective Automatic Method Selection for Nonlinear Regression Modeling.}
International Journal of Neural Systems. Roč. 31, č. 10 (2021), paper no. 2150020. ISSN 0129-0657.

\noindent
P. Vidnerov\'a, R. Neruda.  {\em Vulnerability of classifiers to evolutionary
  generated adversarial examples.}  Neural Networks. Volume 127, July 2020,
p. 168-181. ISSN 0893-6080.

\noindent
S. Slu\v{s}n\'y, R. Neruda, P. Vidnerov\'a.  {\em Comparison of
  Behavior-based and Planning Techniques on the Small Robot Maze
  Exploration Problem.}  Neural Networks. Volume 23, Issue 4 (2010),
p. 560-567. ISSN 0893-6080.

\noindent
R. Neruda, P. Kudov\'a.  {\em Learning Methods for Radial Basis
  Functions Networks.}  Future Generation Computer
Systems. 21. (2005), p. 1131-1142. ISSN 0167-739X

\section{Software}
\begin{tabularx}{0.97\linewidth}{>{\raggedleft}p{2cm}X}
  rbf\_keras & Implementation of an RBF layer for the Keras library. \\
   & Available at
  \href{https://github.com/PetraVidnerova/rbf_keras}{https://github.com/PetraVidnerova/rbf\_keras}\\
  & (12 citations accoding to GoogleScholar, 136 Github stars)\\
&  \\
  Model M & Multiagent epidemic model. One of the key developers. \\
  & Available at
  \href{https://github.com/epicity-cz/model-m}{https://github.com/epicity-cz/model-m} 

\end{tabularx}

\section{Selected Talks}
    {\em From perceptron to deep neural networks}, 2019,
    Workshop Teorie a praxe statistického zpracování dat,
    Palacký University Olomouc,  Nová Seninka.

\noindent
    {\em Adversarial examples - vulnerability of machine learning
      methods and prevention}, 2018, Seminar of the Institute of
    Information Theory and Automation of the Czech Academy of
    Sciences, Prague.

\noindent
    {\em Evolving Architectures of Deep Neural Networks}, 2018,
 Machine Learning and Modelling Seminar, The Faculty
of Mathematics and Physics, Charles University, Prague.

\noindent
{\em Evolution of Composite Kernel Functions for Regularization
  Networks}, 2011, Machine Learning and Modelling Seminar, The Faculty
of Mathematics and Physics, Charles University, Prague.

\noindent
{\em Hybrid learning methods in Bang and Regularization Networks},
 2005, department seminar at University of Edinburgh, UK.

\section{Popularization}
Talk {\em Model M - an agent based epidemiological model}, at the
BISOP book launch event, 2023.

\noindent
Talk in Czech {\em Umělá inteligence: dobrý sluha, zlý pán?}, Open
Day, Institute of Computer Science, The Czech Academy of Sciences,
2019.

\noindent
Talk in Czech {\em Hluboké neuronové sítě}, Open
Day, Institute of Computer Science, The Czech Academy of Sciences,
2017.

\noindent
Joint talk with Roman Neruda at the seminar for high school teachers,
Nové Hrady, 2008.

\section{Community Service}

\begin{tabularx}{0.97\linewidth}{>{\raggedleft}p{2cm}X}
\gray  professional & PC member, reviewer \\
    & member of conference programme committees:
 AIAI 2016, AIAI 2018-2023, EANN 2015-2023, EML GECCO 2016-2023, IJCNN 2017, IJCNN 2019-2023,
ICANN 2018, ICANN 2023, ICONIP 2023, ITAT 2009 \\
    & reviewing for scientific journals: Neural Processing Letters,
    IEEE Transactions on Cybernetics, Computing and Informatics, IEEE
    Transactions on Evolutionary Computations, Neural Networks, Natural
    Computing, Analytical Letters, IEEE Transactions on Neural Networks
    and Learning Systems, Computer Science Review, IEEE Sensors Journal,
    Computers \& security; reviewer for GA UK \\
%\gray other & \\
  &  \\
    \gray & working as a Scientific Secretary of  Institute of Computer Science (since 2023)\\
    & taking care of the \href{http://zatisi.cs.cas.cz}{blog} of Institute of Computer Science (since 2015) \\

    &  \\

    \gray & BISOP, scientific board member (since 2020) \\
     & \href{http://bisop.cz}{http://bisop.cz} \\
  &   \\
\gray free-time & teaching at PyLadies.cz courses (since 2018) \\
& PyLadies is a community of female Python programmers helping women to get
familiar with IT. \\
\gray & author of machine learning study materials for   
data analysis course organised by PyLadies \&  PyData community (2020).
\end{tabularx}


\section{Languages}
\begin{tabularx}{0.97\linewidth}{>{\raggedleft}p{2cm}X}
  Czech &  native \\
  English & C1 (CAE certificate, 2006) \\
  German & elementary \\ 
\end{tabularx}

\section{Other skills}
\begin{tabularx}{0.97\linewidth}{>{\raggedleft}p{2.5cm}X}
  \gray  Programming Languages &  Python, bash (in past: Pascal, C/C++, MPI, Perl, PHP, SQL, JavaScript),
   basic knowledge of HTML and CSS\\
   & familiar with Python libraries: numpy, pandas, matplotlib, seaborn, scikit-learn,
   Keras, Tensorflow, Pytorch\\
%   & moderate experience with Django framework \\
   & \href{http://www.cs.cas.cz/~petra/cv/cert/certificate_intel.pdf}{AI Intel certificate}\\
\gray  Other & LaTeX, git, enthusiastic Linux user \\
%\gray   & enthusiastic Linux user \\
\end{tabularx}


%% \section{Miscellaneous}
%% {\small
%% \begin{tabularx}{0.97\linewidth}{>{\raggedleft}p{2.5cm}X}
%%   certificates & First Aid Kids - Basic Life Support. SHOCK Training Center. (2011) \\
%%   & {\em Respektovat a b\'yt respektov\'an.} Course of communication
%%   skills focused on a parent-child communication. Accredited by M\v{S}MT (Ministry of Education, Youth and Sports). (2013) \\
%%   past experience & Leadership of programming hobby courses for secondary school children (Pascal, C). (1996-2001) \\
%% \end{tabularx}
%% }

\pagebreak

\noindent
{\addfontfeature{LetterSpace=20.0}\fontsize{18}{18}\selectfont\scshape Full List of Publications} 

\section{Journal papers}
Berec, Luděk -  Diviák,  Tomáš - Kuběna, Aleš - Levínský, René - Neruda, Roman - Suchopárová, Gabriela - Šlerka,  Josef - Šmíd, Martin - Trnka, Jan - Tuček, Vít - Vidnerová, Petra - Zajíček, Milan,
{\em On the contact tracing for COVID-19: A simulation study}, 
Epidemics, Volume 43, (2023), ISSN 1755-4365.

\vspace{0.4em}
\noindent
Berec, Luděk - Smyčka, J. - Levínský, René - Hromádková, Eva - Šoltés, Michal - Šlerka, J. - Tuček, V. - Trnka, J. - Šmíd, Martin - Zajíček, Milan - Diviák, T. - Neruda, Roman - Vidnerová, Petra, {\em Delays, Masks, the Elderly, and Schools: First Covid-19 Wave in the Czech Republic}, Bulletin of Mathematical Biology,  84, no. 8 (2022), ISSN 0092-8240

\vspace{0.4em}
\noindent
Berec, Luděk - Levínský, R. - Weiner, J. - Šmíd, Martin - Neruda, Roman - Vidnerová, Petra - Suchopárová, Gabriela, {\em Importance of vaccine action and availability and epidemic severity for delaying the second vaccine dose}, Scientific Reports,  12, no. 1 (2022), ISSN 2045-2322

\vspace{0.4em}
\noindent
Kalina, Jan - Vidnerová, Petra, {\em Least Weighted Squares Quantiles Reveal How Competitiveness Contributes to Tourism Performance}, Finance a úvěr-Czech Journal of Economics and Finance,  72, no. 2 (2022), pages 150-171, ISSN 0015-1920

\vspace{0.4em}
\noindent
Kalina, Jan - Neoral, Aleš - Vidnerová, Petra, {\em Effective Automatic Method Selection for Nonlinear Regression Modeling}, International Journal of Neural Systems,  31, no. 10 (2021), ISSN 0129-0657

\vspace{0.4em}
\noindent
Vidnerová, Petra - Neruda, Roman, {\em Air Pollution Modelling by Machine Learning Methods}, Modelling,  2, no. 4 (2021), pages 659-674

\vspace{0.4em}
\noindent
Vidnerová, Petra - Neruda, Roman, {\em Vulnerability of classifiers to evolutionary generated adversarial examples}, Neural Networks,  127, July (2020), pages 168-181, ISSN 0893-6080

\vspace{0.4em}
\noindent
Vidnerová, Petra - Neruda, Roman, {\em Kernel Function Tuning for Single-Layer Neural Networks}, International Journal of Machine Learning and Computing,  8, no. 4 (2018), pages 354-360, ISSN 2010-3700

\vspace{0.4em}
\noindent
Slušný, Stanislav - Neruda, Roman - Vidnerová, Petra, {\em Comparison of Behavior-based and Planning Techniques on the Small Robot Maze Exploration Problem}, Neural Networks,  23, no. 4 (2010), pages 560-567, ISSN 0893-6080

\vspace{0.4em}
\noindent
Neruda, Roman - Vidnerová, Petra, {\em Learning Errors by Radial Basis Function Neural Networks and  Regularization Networks}, International Journal of  Grid and Distributed Computing,  1, no. 2 (2009), pages 49-57, ISSN 2005-4262

\vspace{0.4em}
\noindent
Neruda, Roman - Kudová, Petra, {\em Learning Methods for Radial Basis Functions Networks}, Future Generation Computer Systems,  21, - (2005), pages 1131-1142, ISSN 0167-739X

\vspace{0.4em}
\noindent
Neruda, Roman - Kudová, Petra, {\em Hybrid Learning of RBF Networks}, Neural Network World,  12, no. 6 (2002), pages 573-585, ISSN 1210-0552

\section{Chapters}
Vidnerová, Petra - Suchopárová, Gabriela - Neruda, Roman, {\em Simulace epidemiologických opatření v multiagentním modelu}, Rok s pandemií covid-19, pages 87-96, ISBN 978-80-246-5273-3

\vspace{0.4em}
\noindent
Kalina, Jan - Vidnerová, Petra - Soukup, Lubomír, {\em Modern Approaches to Statistical Estimation of Measurements in the Location Model and Regression}, Handbook of Metrology and Applications, First Online: 23 July 2022, ISBN 978-981-19-1550-5


\section{In conference proceedings}
Kalina, Jan - Vidnerová, Petra, {\em Properties of the weighted and robust implicitly weighted correlation coefficients}, Proceedings of the ICANN 2023 Thirty-Second International Conference on Artificial Neural Networks (INPRINT)

\vspace{0.4em}
\noindent
Šíma, Jiří - Vidnerová, Petra - Mrázek, V., {\em Energy Complexity Model for Convolutional Neural Networks (to appear)}, Proceedings of the ICANN 2023 Thirty-Second International Conference on Artificial Neural Networks (INPRINT)

\vspace{0.4em}
\noindent
Vidnerová, Petra - Kalina, Jan, {\em Multi-objective Bayesian Optimization for Neural Architecture Search}, Artificial Intelligence and Soft Computing. 21st International Conference, ICAISC 2022. Proceedings, Part I, pages 144-153, ISBN 978-3-031-23491-0, ISSN 0302-9743

\vspace{0.4em}
\noindent
Suchopárová, Gabriela - Vidnerová, Petra - Neruda, Roman - Šmíd, Martin, {\em Using a Deep Neural Network in a Relative Risk Model to Estimate Vaccination Protection for COVID-19}, Engineering Applications of Neural Networks, pages 310-320, ISBN 978-3-031-08222-1, ISSN 1865-0929

\vspace{0.4em}
\noindent
Kalina, Jan - Vidnerová, Petra - Janáček, Patrik, {\em Sparse Versions of Optimized Centroids}, 2022 International Joint Conference on Neural Networks (IJCNN) Proceedings, pages 1-7, ISBN 978-1-7281-8671-9

\vspace{0.4em}
\noindent
Kalina, Jan - Vidnerová, Petra, {\em Application Of Implicitly Weighted Regression Quantiles: Analysis Of The 2018 Czech Presidential Election}, RELIK 2021. Conference Proceedings, pages 332-341, ISBN 978-80-245-2429-0

\vspace{0.4em}
\noindent
Kalina, Jan - Vidnerová, Petra - Tichavský, Jan, {\em A Comparison of Trend Estimators under Heteroscedasticity}, Artificial Intelligence and Soft Computing. ICAISC 2021 Proceedings, Part I, pages 89-98, ISBN 978-3-030-87985-3, ISSN 0302-9743

\vspace{0.4em}
\noindent
Kalina, Jan - Vidnerová, Petra, {\em On kernel-based nonlinear regression estimation}, The 15th International Days of Statistics and Economics Conference Proceedings, pages 450-459, ISBN 978-80-87990-25-4

\vspace{0.4em}
\noindent
Vidnerová, Petra - Neruda, Roman - Suchopárová, Gabriela - Berec, L. - Diviák, T. - Kuběna, Aleš Antonín - Levínský, René - Šlerka, J. - Šmíd, Martin - Trnka, J. - Tuček, V. - Vrbenský, Karel - Zajíček, Milan, {\em Simulation of non-pharmaceutical interventions in an agent based epidemic model}, Proceedings of the 21st Conference Information Technologies – Applications and Theory (ITAT 2021), pages 263-268, ISSN 1613-0073

\vspace{0.4em}
\noindent
Vidnerová, Petra - Kalina, Jan - Güney, Y., {\em A Comparison of Robust Model Choice Criteria Within a Metalearning Study}, Analytical Methods in Statistics, pages 125-141, ISBN 978-3-030-48813-0

\vspace{0.4em}
\noindent
Vidnerová, Petra - Neruda, Roman, {\em Multi-objective Evolution for Deep Neural Network Architecture Search}, Neural Information Processing. ICONIP 2020 Proceedings, Part III, pages 270-281, ISBN 978-3-030-63835-1, ISSN 0302-9743

\vspace{0.4em}
\noindent
Vidnerová, Petra - Procházka, Štěpán - Neruda, Roman, {\em Multiobjective Evolution for Convolutional Neural Network Architecture Search}, Artificial Intelligence and Soft Computing. ICAISC 2020 Proceedings, Part I, pages 261-270, ISBN 978-3-030-61400-3, ISSN 0302-9743

\vspace{0.4em}
\noindent
Vidnerová, Petra - Kalina, Jan, {\em Least Weighted Absolute Value Estimator with an Application to Investment Data}, The 14th International Days of Statistics and Economics Conference Proceedings, pages 1357-1366, ISBN 978-80-87990-22-3

\vspace{0.4em}
\noindent
Kalina, Jan - Vidnerová, Petra, {\em Robust Multilayer Perceptrons: Robust Loss Functions and Their Derivatives}, Proceedings of the 21st EANN (Engineering Applications of Neural Networks) 2020 Conference, pages 546-557, ISBN 978-3-030-48790-4, ISSN 2661-8141

\vspace{0.4em}
\noindent
Kalina, Jan - Vidnerová, Petra, {\em Regression Neural Networks with a Highly Robust Loss Function}, Analytical Methods in Statistics, pages 17-29, ISBN 978-3-030-48813-0

\vspace{0.4em}
\noindent
Kalina, Jan - Vidnerová, Petra, {\em Regression for High-Dimensional Data: From Regularization to Deep Learning}, The 14th International Days of Statistics and Economics Conference Proceedings, pages 418-427, ISBN 978-80-87990-22-3

\vspace{0.4em}
\noindent
Kalina, Jan - Vidnerová, Petra, {\em On Robust Training of Regression Neural Networks}, Functional and High-Dimensional Statistics and Related Fields, pages 145-152, ISBN 978-3-030-47755-4, ISSN 1431-1968

\vspace{0.4em}
\noindent
Kalina, Jan - Vidnerová, Petra, {\em A Metalearning Study for Robust Nonlinear Regression}, Proceedings of the 21st EANN (Engineering Applications of Neural Networks) 2020 Conference, pages 499-510, ISBN 978-3-030-48790-4, ISSN 2661-8141

\vspace{0.4em}
\noindent
Kalina, Jan - Vidnerová, Petra, {\em Implicitly weighted robust estimation of quantiles in linear regression}, Conference Proceedings. 37th International Conference on Mathematical Methods in Economics 2019, pages 25-30, ISBN 978-80-7394-760-6

\vspace{0.4em}
\noindent
Kalina, Jan - Vidnerová, Petra, {\em Robust Training of Radial Basis Function Neural Networks}, Artificial Intelligence and Soft Computing. Proceedings, Part I, pages 113-124, ISBN 978-3-030-20911-7, ISSN 0302-9743

\vspace{0.4em}
\noindent
Vidnerová, Petra - Neruda, Roman, {\em Asynchronous Evolution of Convolutional Networks}, ITAT 2018: Information Technologies – Applications and Theory. Proceedings of the 18th conference ITAT 2018, pages 80-85, ISSN 1613-0073

\vspace{0.4em}
\noindent
Vidnerová, Petra - Neruda, Roman, {\em Deep Networks with RBF Layers to Prevent Adversarial Examples}, Artificial Intelligence and Soft Computing, pages 257-266, ISBN 978-3-319-91252-3, ISSN 0302-9743

\vspace{0.4em}
\noindent
Vidnerová, Petra - Neruda, Roman, {\em Evolving Keras Architectures for Sensor Data Analysis}, Proceedings of the 2017 Federated Conference on Computer Science and Information Systems, pages 109-112, ISBN 978-83-946253-7-5, ISSN 2300-5963

\vspace{0.4em}
\noindent
Vidnerová, Petra - Neruda, Roman, {\em Evolution Strategies for Deep Neural Network Models Design}, Proceedings ITAT 2017: Information Technologies - Applications and Theory, pages 159-166, ISBN 978-1974274741, ISSN 1613-0073

\vspace{0.4em}
\noindent
Vidnerová, Petra - Neruda, Roman, {\em Sensor Data Air Pollution Prediction by Kernel Models}, Proceedings og the 16th IEEE/ACM International Symposium on Cluster, Cloud, and Grid Computing, pages 666-673, ISBN 978-1-5090-2453-7

\vspace{0.4em}
\noindent
Vidnerová, Petra - Neruda, Roman, {\em Evolutionary Generation of Adversarial Examples for Deep and Shallow Machine Learning Models}, Proceedings of the The 3rd Multidisciplinary International Social Networks Conference on SocialInformatics 2016, Data Science 2016, ISBN 978-1-4503-4129-5

\vspace{0.4em}
\noindent
Vidnerová, Petra - Neruda, Roman, {\em Vulnerability of Machine Learning Models to Adversarial Examples}, Proceedings ITAT 2016: Information Technologies - Applications and Theory, pages 187-194, ISBN 978-1-5370-1674-0, ISSN 1613-0073

\vspace{0.4em}
\noindent
Vidnerová, Petra - Neruda, Roman, {\em Product Multi-kernels for Sensor Data Analysis}, Artificial Intelligence and Soft Computing, pages 123-133, ISBN 978-3-319-19323-6, ISSN 0302-9743

\vspace{0.4em}
\noindent
Vidnerová, Petra - Neruda, Roman, {\em Meta-Parameters of Kernel Methods and Their Optimization}, ITAT 2014. Information Technologies - Applications and Theory. Part II, pages 99-105, ISBN 978-80-87136-19-5

\vspace{0.4em}
\noindent
Vidnerová, Petra - Neruda, Roman, {\em Evolving Sum and Composite Kernel Functions for Regularization Networks}, Adaptive and Natural Computing Algorithms. Part I, pages 180-189, ISBN 978-3-642-20281-0, ISSN 0302-9743

%% \vspace{0.4em}
%% \noindent
%% Vidnerová, Petra - Neruda, Roman, {\em Evolution of Product Kernels for Regularization  Networks}, Advanced Intelligent Computing, nevyšlo tiskem, ISBN 978-3-642-24727-9, ISSN 0302-9743

\vspace{0.4em}
\noindent
Vidnerová, Petra - Neruda, Roman, {\em Evolutionary learning of regularization networks with product kernel units}, Systems, Man and Cybernetics, pages 638-643, ISBN 978-1-4577-0652-3, ISSN 1062-922X

\vspace{0.4em}
\noindent
Vidnerová, Petra - Neruda, Roman, {\em Evolutionary Learning of Regularization Networks with Multi-kernel Units}, Advances in  Neural Networks –  ISNN 2011. Part I, pages 538-546, ISBN 978-3-642-21104-1, ISSN 0302-9743

\vspace{0.4em}
\noindent
Neruda, Roman - Vidnerová, Petra, {\em Memetic Evolutionary Learning for Local Unit Networks}, Advances in Neural Networks – ISNN 2010, pages 534-541, ISBN 978-3-642-13277-3, ISSN 0302-9743

\vspace{0.4em}
\noindent
Neruda, Roman - Vidnerová, Petra, {\em Genetic Algorithm with Species for Regularization Network Metalearning}, Advances in Information Technology, pages 192-201, ISBN 978-3-642-16698-3, ISSN 1865-0929

\vspace{0.4em}
\noindent
Vidnerová, Petra - Neruda, Roman, {\em Genetic Algorithm with Species for Regularization Network Metalearning}, Informačné Technológie - Aplikácie a Teória, pages 111-116, ISBN 978-80-970179-3-4

\vspace{0.4em}
\noindent
Vidnerová, Petra - Neruda, Roman, {\em Hybrid Learning of Regularization Neural Networks}, Artificial Intelligence and Soft Computing, pages 124-131, ISBN 978-3-642-13231-5, ISSN 0302-9743

\vspace{0.4em}
\noindent
Slušný, Stanislav - Neruda, Roman - Vidnerová, Petra, {\em Localization with a Low-Cost Robot}, Information Technologies - Applications and Theory, pages 77-80, ISBN 978-80-970179-2-7

\vspace{0.4em}
\noindent
Vidnerová, Petra - Slušný, Stanislav - Neruda, Roman, {\em Evolutionary Trained Radial Basis Function Networks for Robot Control}, Control, Automation, Robotics and Vision, pages 833-838, ISBN 978-1-4244-2286-9

\vspace{0.4em}
\noindent
Vidnerová, Petra - Slušný, Stanislav - Neruda, Roman, {\em Emergence chování robotických agentů: neuroevoluce}, Kognice a umělý život VIII, pages 363-369, ISBN 978-80-7248-462-1

\vspace{0.4em}
\noindent
Vidnerová, Petra - Neruda, Roman, {\em Testing Error Estimates for Regularization and Radial Function Networks}, Advances in Neural Networks - ISNN 2008, pages 549-554, ISBN 978-3-540-87731-8

\vspace{0.4em}
\noindent
Gemrot, J. - Kadlec, R. - Brom, C. - Vidnerová, Petra, {\em Evoluce chování agentů v 3D prostředí}, Informačné technológie - Aplikácie a teória, pages 3-10, ISBN 978-80-969184-8-5

\vspace{0.4em}
\noindent
Slušný, Stanislav - Neruda, Roman - Vidnerová, Petra, {\em Comparison of RBF Network Learning and Reinforcement Learning on the Maze Exploration Problem}, Artificial Neural Networks - ICANN 2008, pages 720-729, ISBN 978-3-540-87535-2

\vspace{0.4em}
\noindent
Slušný, Stanislav - Neruda, Roman - Vidnerová, Petra, {\em Learning Algorithms for Small Mobile Robots: Case Study on Maze Exploration}, Information Technologies - Applications and Theory, pages 49-54, ISBN 978-80-969184-9-2

\vspace{0.4em}
\noindent
Slušný, Stanislav - Vidnerová, Petra - Neruda, Roman, {\em Emergencia chovania robotických agentov: učenie posilovaním}, Kognice a umělý život VIII, pages 295-299, ISBN 978-80-7248-462-1

\vspace{0.4em}
\noindent
Neruda, Roman - Slušný, Stanislav - Vidnerová, Petra, {\em Two Learning Approaches to Maze Exploration: Case Study with E-puck Mobile Robots}, World Congress on Engineering and Computer Science, pages 655-660, ISBN 978-988-98671-0-2

\vspace{0.4em}
\noindent
Neruda, Roman - Vidnerová, Petra, {\em Supervised Learning Errors by Radial Basis Function Neural Networks and Regularization Networks}, Proceedings of Second International Conference on Future Generation Communication and Networking Symposia, pages 193-196, ISBN 978-1-4244-3430-5

\vspace{0.4em}
\noindent
Neruda, Roman - Slušný, Stanislav - Vidnerová, Petra, {\em Performance Comparison of Relational Reinforcement Learning and RBF Neural Networks for Small Mobile Robots}, Proceedings of Second International Conference on Future Generation Communication and Networking Symposia, pages 29-32, ISBN 978-1-4244-3430-5

\vspace{0.4em}
\noindent
Slušný, Stanislav - Neruda, Roman - Vidnerová, Petra, {\em Rule-based Analysis of Behaviour Learned by Evolutionary and Reinforcement Algorithms}, Advanced Intelligent Computing Theories and Applications With Aspects of Artificial Intelligence, pages 284-291, ISBN 978-3-540-85983-3

\vspace{0.4em}
\noindent
Neruda, Roman - Slušný, Stanislav - Vidnerová, Petra, {\em Evolution of Simple Behavior Patterns for Autonomous Robotic Agent}, System Science and Simulation in Engineering, pages 411-417, ISBN 978-960-6766-14-5

\vspace{0.4em}
\noindent
Kudová, Petra, {\em Clustering Genetic Algorithm}, Database and Expert Systems Applications, pages 138-142, ISBN 978-0-7695-2932-5

\vspace{0.4em}
\noindent
Kudová, Petra, {\em Clustering using Genetic Algorithms}, Evolutionary Techniques in Data-processing, pages 1-11, ISBN 978-80-248-1332-5

\vspace{0.4em}
\noindent
Vidnerová, Petra, {\em Learning with Regularization Networks Mixtures}, Information Technologies - Applications and Theory, pages 109-113, ISBN 978-80-969184-7-8

\vspace{0.4em}
\noindent
Slušný, Stanislav - Vidnerová, Petra - Neruda, Roman, {\em Testing Different Evolutionary Neural Networks for Autonomous Robot Control}, Information Technologies - Applications and Theory, pages 103-108, ISBN 978-80-969184-7-8

\vspace{0.4em}
\noindent
Slušný, Stanislav - Vidnerová, Petra - Neruda, Roman, {\em Behavior Emergence in Autonomous Robot control by Means of Feedforward and Reccurent Neural Networks}, WCECS 2007, pages 518-523, ISBN 978-988-98671-6-4

\vspace{0.4em}
\noindent
Slušný, Stanislav - Neruda, Roman - Vidnerová, Petra, {\em Behaviour Patterns Evolution on Individual and Group Level}, Computational Intelligence, Man-Machine Systems and Cybernetics, pages 24-29, ISBN 978-960-6766-21-3

\vspace{0.4em}
\noindent
Řezanková, H. - Húsek, Dušan - Kudová, Petra - Snášel, V., {\em Comparison of some Approaches to Clustering Categorical Data}, Proceedings in Computational Statistics, pages 607-613, ISBN 3-7908-1708-2

\vspace{0.4em}
\noindent
Kudová, Petra, {\em The Role of Kernel Function in Regularization Network}, Information Technologies - Applications and Theory, pages 101-105, ISBN 80-969184-4-3

\vspace{0.4em}
\noindent
Kudová, Petra - Šámalová, Terezie, {\em Sum and Product Kernel Regularization Networks}, Artificial Intelligence and Soft Computing – ICAISC 2006, pages 56-65, ISBN 3-540-35748-3

\vspace{0.4em}
\noindent
Kudová, Petra - Řezanková, H. - Húsek, Dušan - Snášel, V., {\em Categorical Data Clustering using Statistical Methods and Neural Networks}, Spring Young Researches' Colloquium on Database and Information Systems, pages 19-23

\vspace{0.4em}
\noindent
Neruda, M. - Neruda, Roman - Kudová, Petra, {\em Forecasting Runoff with Artificial Neural Networks}, Progress in Surface and Subsurface Water Studies at Plot and Small Basin Scale, pages 65-69

\vspace{0.4em}
\noindent
Kudová, Petra - Šámalová, Terezie, {\em Product Kernel Regularization Networks}, Adaptive and Natural Computing Algorithms, pages 433-436, ISBN 3-211-24934-6

\vspace{0.4em}
\noindent
Kudová, Petra, {\em Learning with Kernel Based Regularization Methods}, ITAT 2005. Information Technologies - Applications and Theory, pages 83-92, ISBN 80-7097-609-8

\vspace{0.4em}
\noindent
Kudová, Petra - Neruda, Roman, {\em Kernel Based Learning Methods: Regularization Networks and RBF Networks}, Deterministic and Statistical Methods in Machine Learning, pages 124-136, ISBN 3-540-29073-7

\vspace{0.4em}
\noindent
Kudová, Petra, {\em Comparison of Kernel Based Regularization Networks and RBF Networks}, ITAT 2004. Information Technologies - Applications and Theory, pages 59-68, ISBN 80-7097-589-X

\vspace{0.4em}
\noindent
Neruda, Roman - Krušina, Pavel - Kudová, Petra - Rydvan, Pavel - Beuster, G., {\em Bang 3: A Computational Multi-Agent System}, Intelligent Agent Technology, pages 563-564, ISBN 0-7695-2101-0

\vspace{0.4em}
\noindent
Neruda, Roman - Kudová, Petra, {\em Hybrid Learning of RBF Networks}, Computational Science, pages 594-603, ISBN 3-540-43594-8, ISSN 0302-9743

\vspace{0.4em}
\noindent

\section{Other}
VIDNEROVÁ, Petra. {\em Hitoshi Iba: Evolutionary approach to machine learning and deep neural networks: neuro-evolution and gene regulatory networks (review)}, Genetic Programming and Evolvable Machines. 2019, 20(2), 151-153, ISSN 1389-2576, E-ISSN 1573-7632 

\section{Software}
Coufal, David - Hakl, František - Vidnerová, Petra, {\em General-purpose Library of ML/AI Methods}, 2022, \href{https://github.com/PetraVidnerova/nck_python}{https://github.com/PetraVidnerova/nck\_python}

\vspace{0.4em}
\noindent
Coufal, David - Hakl, František - Vidnerová, Petra, {\em General-purpose Library of ML/AI Methods for CUDA Cores}, 2022, \href{https://github.com/PetraVidnerova/nck_matlab}{ https://github.com/PetraVidnerova/nck\_matlab}

\vspace{0.4em}
\noindent
Berec, Luděk - Diviák, T. - Kuběna, Aleš Antonín - Levínský, René - Neruda, Roman - Suchopárová, Gabriela - Šlerka, J. - Šmíd, Martin - Trnka, Jan - Tuček, Vít - Vidnerová, Petra - Vrbenský, Karel - Zajíček, Milan - Zapletal, František, {\em Epicity}, 2021, \href{https://github.com/epicity-cz/model-m/releases/tag/v1.0}{https://github.com/epicity-cz/model-m/releases/tag/v1.0}

\vspace{0.4em}
\noindent
Coufal, David - Hakl, František - Vidnerová, Petra, {\em General-purpose algorithms for machine learning}, 2020, \href{https://github.com/epicity-cz/model-m/releases/tag/v1.0}{https://github.com/epicity-cz/model-m/releases/tag/v1.0}

\vspace{0.4em}
\noindent
Kalina, Jan - Vidnerová, Petra - Peštová, Barbora, {\em Metalearning for robust regression 1.0}, 2020, \href{https://github.com/jankalinaUI/Metalearning-for-robust-regression}{https://github.com/jankalinaUI/Metalearning-for-robust-regression}

\vspace{0.4em}
\noindent
Jurica, Tomáš - Vidnerová, Petra - Kalina, Jan, {\em Robust interquantile training of neural networks}, 2019, \href{https://github.com/jankalinaUI/Quantile}{https://github.com/jankalinaUI/Quantile}

\vspace{0.4em}
\noindent
Vidnerová, Petra, {\em RBF-Keras: an RBF Layer for Keras Library}, 2019,\newline \href{https://github.com/PetraVidnerova/rbf_keras}{https://github.com/PetraVidnerova/rbf\_keras}

\vspace{0.4em}
\noindent



\pagebreak

\noindent
{\addfontfeature{LetterSpace=20.0}\fontsize{18}{18}\selectfont\scshape Description of selected results} 

\section{Learning Algorithms for RBF Networks}
In the 1990s and early 21st century, the field of machine learning
was dominated by the so-called kernel methods. Among these, we can
include the highly popular Radial Basis Function Networks (RBF
networks).

As part of my doctoral thesis, I delved into these methods. One of the
primary outcomes [1] was the proposal of several algorithms for
training RBF networks and an analysis of these algorithms. All methods
were implemented and experiments were conducted to demonstrate the
relationship between the training time and the accuracy of the
results.

\centerline{}
\noindent
[1] R. Neruda, P. Kudová. Learning Methods for Radial Basis Functions
Networks. Future Generation Computer Systems. 21. (2005),
p. 1131-1142. ISSN 0167-739X
\href{https://dl.acm.org/doi/10.5555/1088377.1708275}{https://dl.acm.org/doi/10.5555/1088377.1708275}


\section{Generating of Adversarial Examples and Their Analysis}

In recent years, the question of the security and reliability of
artificial intelligence algorithms has gained significant
attention. This is closely related to the concept of "adversarial"
patterns, which are intentionally created patterns designed to deceive
a given classification model. For instance, in image classification,
an adversarial pattern may appear indistinguishable from the original
image to the human eye, yet a properly trained model may classify it
incorrectly.

In our article [2], we have contributed to the understanding
of adversarial patterns, particularly emphasizing that they are not
limited to neural networks but apply to a wide range of classifiers,
from SVMs to decision trees. To address this issue, we have introduced
our own evolutionary algorithm for adversarial
patterns crafting. Its notable feature is its "black-box" approach, meaning it
doesn't require knowledge of the model's internal settings and
structure. As a result, it can be employed to undermine (in the
context of adversarial patterns) virtually any classification model.

Furthermore, through our experiments, we have demonstrated that
certain adversarial patterns can be transferred between models,
particularly when dealing with models of similar nature (e.g., two
neural networks with similar architectures).

Additionally, we have shown that networks equipped with local units
(e.g., RBF networks) exhibit higher resilience against adversarial
patterns.

\centerline{}
\noindent
[2] P. Vidnerová, R. Neruda. Vulnerability of classifiers to
evolutionary generated adversarial examples, Neural Networks, Volume
127, 2020.\newline
\href{https://doi.org/10.1016/j.neunet.2020.04.015}{https://doi.org/10.1016/j.neunet.2020.04.015}


\section{Epidemic Modeling - Multi-agent System Model M}
In 2020, I joined a group of scientists focused on modeling the spread
of the COVID-19 disease. One of the key outcomes is a multi-agent
model named Model M [3], along with software [4] that enables
experimentation with the model.

The model operates with a realistic network graph and allows for
simulating a wide range of interventions, such as global contact
restrictions, quarantines, and contact tracing.

The model is designed modularly, allowing for diverse application
deployments. The model was used in a study on virus transmission in
schools in collaboration with the Ministry of Education, Youth, and
Sports of the Czech Republic.

I contributed to the model's design, implementation, and experiments.

\centerline{}
\noindent
[3] Berec, et al. On the Contact Tracing for COVID-19: A simulation
study. Epidemics, Volume 43, (2023), ISSN 1755-4365.
\href{https://doi.org/10.1016/j.epidem.2023.100677}{https://doi.org/10.1016/j.epidem.2023.100677}

\noindent
[4] Berec, et al. {\em Epicity}, 2021,
\href{https://github.com/epicity-cz/model-m/releases/tag/v1.0}{https://github.com/epicity-cz/model-m/releases/tag/v1.0}


\end{document}
